\documentclass[12pt]{article}
\usepackage{amsmath}
\usepackage[margin = 1in]{geometry}
\usepackage{graphicx}
\usepackage{booktabs}
\usepackage{natbib}
\usepackage{lipsum}
\usepackage[colorlinks=true, citecolor=blue]{hyperref}
\title{Machine Learning for Handwriting Recognition: Algorithms and Techniques}
\author{Lai Jiang\\
  University of Connecticut Department of Statistics
}

\begin{document}
\maketitle

\paragraph{Introduction}
In an era where digital communication pervades nearly every facet of our lives, the beauty and uniqueness of human handwriting still holds profound significance. From historical manuscripts to personal notes, handwriting offers a glimpse into a person's individuality and emotions. Yet, deciphering diverse handwritten styles and converting them into digital text is a challenging task. The irregularities, variations, and idiosyncrasies present in handwriting have made its recognition one of the most intriguing problems in the realm of pattern recognition and artificial intelligence. However, the evolution of machine learning, with its robust algorithms and sophisticated techniques, has brought groundbreaking advancements to this domain. Handwriting recognition, once a distant dream, is now manifesting in various applications ranging from mail sorting to bank cheque processing and beyond. This essay delves into the heart of machine learning algorithms and techniques that power handwriting recognition, shedding light on the innovations that bridge the gap between the analog strokes of a pen and the digital bytes of a computer.

\paragraph{Paragraph 1}


\paragraph{Paragraph 2}


\paragraph{Paragraph 3}


\paragraph{Paragraph 4}


\bibliographystyle{plain}
\bibliography{Citations.bib}
\cite{824821}
\cite{NIPS2008_66368270}
\cite{6981034}
\end{document}