\documentclass[12pt]{article}
\usepackage{amsmath}
\usepackage[margin = 1in]{geometry}
\usepackage{graphicx}
\usepackage{booktabs}
\usepackage{natbib}
\usepackage{lipsum}
\usepackage[colorlinks=true, citecolor=blue]{hyperref}
\title{Machine Learning for Handwriting Recognition: Algorithms and Techniques}
\author{Lai Jiang\\
  University of Connecticut Department of Statistics
}

\begin{document}
\maketitle

\paragraph{Introduction}
In an era where digital communication pervades nearly every facet of our lives, the beauty and uniqueness of human handwriting still holds profound significance. From historical manuscripts to personal notes, handwriting offers a glimpse into a person's individuality and emotions. Yet, deciphering diverse handwritten styles and converting them into digital text is a challenging task. The irregularities, variations, and idiosyncrasies present in handwriting have made its recognition one of the most intriguing problems in the realm of pattern recognition and artificial intelligence. However, the evolution of machine learning, with its robust algorithms and sophisticated techniques, has brought groundbreaking advancements to this domain. Handwriting recognition, once a distant dream, is now manifesting in various applications ranging from mail sorting to bank cheque processing and beyond. This essay delves into the heart of machine learning algorithms and techniques that power handwriting recognition, shedding light on the innovations that bridge the gap between the analog strokes of a pen and the digital bytes of a computer.

\paragraph{Objectives}
This project will mainly focus to the algorithms and techniques of handwriting recognizing in machine learning. I'm going to investigate the normal procedures of how to perform a standard recognition, and elaborate the the principle behind this. I'll also conduct a two dimensional (Pixel by Pixel) hand writing recognition to see their difference between each models, the functions, parameters, and their time complexity.

\paragraph{Data Collections}
This topic belongs to the deep learning section of machine learning so python, Scikit - Learn or PyTorch is required. The data for this project will gather comprehensive variables with various libraries. There are bunch of library that can allow us to do the handwriting recognition but I will select the MNIST and IAM Database since they both contain more than 70000 pictures in the library for training. Moreover, in order to adjust the result that the model produces, we also need to adjust the parameters in the models itself. And the last, find an appropriate activation function is necessary, we're not only 

\paragraph{Research Methodology}
In our quest to understand the intricacies of machine learning applications in handwriting recognition, a multi-faceted research methodology has been adopted. Initially, we identified the core libraries and frameworks that have made significant strides in this domain, including TensorFlow, PyTorch, scikit-learn, and OpenCV. By examining the capabilities of these libraries, we gained insights into the tools that professionals are leveraging for handwriting analysis. We sourced and analyzed datasets, such as the MNIST, IAM Handwriting Database, and CEDAR, to appreciate the diversity of data and challenges faced in preprocessing and feature extraction. With a focus on understanding model architectures and training parameters, an in-depth exploration of Convolutional Neural Networks (CNNs), Recurrent Neural Networks (RNNs), and LSTMs was conducted. To assess the efficiency and accuracy of various models, evaluation metrics like precision, recall, and F1-score were applied, and results were compared. Furthermore, contemporary research papers were reviewed to identify recent advancements and innovative techniques that are pushing the boundaries of what’s possible in handwriting recognition. Throughout this research process, ethical considerations, especially those pertaining to data privacy and potential biases in recognition models, were given paramount importance. This comprehensive methodology ensures that our analysis is both technically profound and contextually relevant, setting the stage for valuable insights and discussions in the realm of machine learning-based handwriting recognition.

\paragraph{Discussion}
In this section, I'll summaize and review what I talked in previous sections, and except the content I mentioned above, It's hypothesized that how could we bring the efficiency even further by optimizing the algorithm or fixing the conciseness of the code. 

\bibliographystyle{plain}
\bibliography{Citations.bib}
\cite{824821}
\cite{NIPS2008_66368270}
\cite{6981034}
\end{document}